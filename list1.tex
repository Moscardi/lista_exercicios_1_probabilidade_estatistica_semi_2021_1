\chapter{Lista de Exercícios 1}

\begin{enumerate}
	\item Nos exercícios abaixo, identifique a população, a unidade elementar, o tamanho da	população, a amostra, a unidade amostral, o tamanho da amostra, o parâmetro de interesse, a técnica de amostragem, a variável de interesse e respectiva unidade de medida, classifique a variável e identifique a escala de medição da variável:
	
	\subitem A gerência de uma loja de eletrônicos deseja verificar se os consumidores que compraram uma televisão de 32 polegadas, nos últimos seis meses, estavam satisfeitos com o produto. A gerência planeja investigar 500 desses consumidores.
	
	\begin{itemize}	
		\item \textbf{População:}~Todos os consumidores da loja que compraram uma televisão de 32 polegadas, nos últimos seis meses
		
		\item \textbf{Unidade elementar:}~Cada consumidor da loja que comprou uma televisão de 32 polegadas, nos
últimos seis meses
		
		\item \textbf{Tamanho da população (N):}~não informado
		
		\item \textbf{Amostra:}~Todos os consumidores da loja investigados pela gerência
		
		\item \textbf{Unidade amostral:}~Cada consumidor da loja investigado pela gerência
		
		\item \textbf{Tamanho da amostra (n):}~500 consumidores
		
		\item \textbf{Parâmetro de interesse $\left(\varTheta\right)$:}~quantidade de consumidores satisfeitos com o produto; percentual de
consumidores satisfeitos com o produto; proporção de consumidores satisfeitos com o produto
		
		\item \textbf{Técnica de amostragem:}~não informado
		
		\item \textbf{Variável:}~Satisfação com o produto (sim, não); Grau de satisfação com o produto (muito satisfeito,
satisfeito, pouco satisfeito)
		
		\item \textbf{Unidade de medida da variável:}~não há
		
		\item \textbf{Classificação da variável:}~nominal, se a variável for "satisfação com o produto"; ordinal, se a variável
for "grau de satisfação com o produto"
		
		\item \textbf{Escala de medição da variável:}~escala nominal, se a variável "satisfação com o produto" tiver resultados
"sim" ou "não"; escala ordinal, se a variável "grau de satisfação com o produto" tiver como resultados
"muito satisfeito", "satisfeito", "pouco satisfeito"; escala de proporcionalidade (ou escala de razão) se a
variável "grau de satisfação com o produto" for um número de 0 a 100, sendo "0" nada satisfeito e "100"
totalmente satisfeito.
	\end{itemize}

	\subitem Numa pesquisa de intenção de voto para tentar prever o resultado de uma eleição para
prefeito, de 32.873 eleitores foram entrevistados 2.560, em 600 domicílios, respeitando-se a proporção de sexo, classe social, idade e bairro da residência dos eleitores.
	
	\begin{itemize}	
		\item \textbf{População:}~
			
		\item \textbf{Unidade elementar:}~
		
		\item \textbf{Tamanho da população (N):}~
		
		\item \textbf{Amostra:}~
		
		\item \textbf{Unidade amostral:}~
		
		\item \textbf{Tamanho da amostra (n):}~
		
		\item \textbf{Parâmetro de interesse $\left(\varTheta\right)$:}~
		
		\item \textbf{Técnica de amostragem:}~
		
		\item \textbf{Variável:}~
		
		\item \textbf{Unidade de medida da variável:}~
		
		\item \textbf{Classificação da variável:}~
		
		\item \textbf{Escala de medição da variável:}~
	\end{itemize}

	\subitem 

	\begin{itemize}	
		\item \textbf{População:}~
		
		\item \textbf{Unidade elementar:}~
		
		\item \textbf{Tamanho da população (N):}~
		
		\item \textbf{Amostra:}~
		
		\item \textbf{Unidade amostral:}~
		
		\item \textbf{Tamanho da amostra (n):}~
		
		\item \textbf{Parâmetro de interesse $\left(\varTheta\right)$:}~
		
		\item \textbf{Técnica de amostragem:}~
		
		\item \textbf{Variável:}~
		
		\item \textbf{Unidade de medida da variável:}~
		
		\item \textbf{Classificação da variável:}~
		
		\item \textbf{Escala de medição da variável:}~
	\end{itemize}
	
\end{enumerate}



