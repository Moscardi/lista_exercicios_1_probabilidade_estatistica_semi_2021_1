\chapter{Lista de Exercícios 1}

\begin{enumerate}
	\item Nos exercícios abaixo, identifique a população, a unidade elementar, o tamanho da	população, a amostra, a unidade amostral, o tamanho da amostra, o parâmetro de interesse, a técnica de amostragem, a variável de interesse e respectiva unidade de medida, classifique a variável e identifique a escala de medição da variável:
	
	\subitem A gerência de uma loja de eletrônicos deseja verificar se os consumidores que compraram uma televisão de 32 polegadas, nos últimos seis meses, estavam satisfeitos com o produto. A gerência planeja investigar 500 desses consumidores.
	
	\begin{itemize}	
		\item \textbf{População:}~Todos os consumidores da loja que compraram uma televisão de 32 polegadas, nos últimos seis meses
		
		\item \textbf{Unidade elementar:}~Cada consumidor da loja que comprou uma televisão de 32 polegadas, nos
últimos seis meses
		
		\item \textbf{Tamanho da população (N):}~não informado
		
		\item \textbf{Amostra:}~Todos os consumidores da loja investigados pela gerência
		
		\item \textbf{Unidade amostral:}~Cada consumidor da loja investigado pela gerência
		
		\item \textbf{Tamanho da amostra (n):}~500 consumidores
		
		\item \textbf{Parâmetro de interesse $\left(\varTheta\right)$:}~quantidade de consumidores satisfeitos com o produto; percentual de
consumidores satisfeitos com o produto; proporção de consumidores satisfeitos com o produto
		
		\item \textbf{Técnica de amostragem:}~não informado
		
		\item \textbf{Variável:}~Satisfação com o produto (sim, não); Grau de satisfação com o produto (muito satisfeito,
satisfeito, pouco satisfeito)
		
		\item \textbf{Unidade de medida da variável:}~não há
		
		\item \textbf{Classificação da variável:}~nominal, se a variável for "satisfação com o produto"; ordinal, se a variável
for "grau de satisfação com o produto"
		
		\item \textbf{Escala de medição da variável:}~escala nominal, se a variável "satisfação com o produto" tiver resultados
"sim" ou "não"; escala ordinal, se a variável "grau de satisfação com o produto" tiver como resultados
"muito satisfeito", "satisfeito", "pouco satisfeito"; escala de proporcionalidade (ou escala de razão) se a
variável "grau de satisfação com o produto" for um número de 0 a 100, sendo "0" nada satisfeito e "100"
totalmente satisfeito.
	\end{itemize}

	\subitem Numa pesquisa de intenção de voto para tentar prever o resultado de uma eleição para
prefeito, de 32.873 eleitores foram entrevistados 2.560, em 600 domicílios, respeitando-se a proporção de sexo, classe social, idade e bairro da residência dos eleitores.
	
	\begin{itemize}	
		\item \textbf{População:}~Eleitores
			
		\item \textbf{Unidade elementar:}~1 eleitor
		
		\item \textbf{Tamanho da população (N):}~$32.873$ eleitores
		
		\item \textbf{Amostra:}~Eleitores entrevistados
		
		\item \textbf{Unidade amostral:}~Cada eleitor entrevistado
		
		\item \textbf{Tamanho da amostra (n):}~$2.560$ eleitores em 600 domicílios
		
		\item \textbf{Parâmetro de interesse $\left(\varTheta\right)$:}~intenção de voto de cada eleitor
		
		\item \textbf{Técnica de amostragem:}~amostragem proporcional estratificada
		
		\item \textbf{Variável:}~intenção de voto
		
		\item \textbf{Unidade de medida da variável:}~voto
		
		\item \textbf{Classificação da variável:}~variável quantitativa discreta
		
		\item \textbf{Escala de medição da variável:}~a escala da variável será ordinal 
	\end{itemize}

	\subitem Um lote contém $2.342$ peças de um mesmo tipo, não ordenadas. Retira-se dali, ao
acaso e sem reposição, 85 peças para estimar o percentual de peças defeituosas no
lote.

	\begin{itemize}	
		\item \textbf{População:}~
		
		\item \textbf{Unidade elementar:}~
		
		\item \textbf{Tamanho da população (N):}~
		
		\item \textbf{Amostra:}~
		
		\item \textbf{Unidade amostral:}~
		
		\item \textbf{Tamanho da amostra (n):}~
		
		\item \textbf{Parâmetro de interesse $\left(\varTheta\right)$:}~
		
		\item \textbf{Técnica de amostragem:}~
		
		\item \textbf{Variável:}~
		
		\item \textbf{Unidade de medida da variável:}~
		
		\item \textbf{Classificação da variável:}~
		
		\item \textbf{Escala de medição da variável:}~
	\end{itemize}

	\subitem A polícia rodoviária de uma localidade deseja estimar a proporção de carros com pneus
carecas que passam pelo posto policial, durante um mês. Assim, adota a prática de
parar um carro para inspeção dos pneus, a cada 30 minutos, durante duas semanas.
	
	\begin{itemize}	
		\item \textbf{População:}~
		
		\item \textbf{Unidade elementar:}~
		
		\item \textbf{Tamanho da população (N):}~
		
		\item \textbf{Amostra:}~
		
		\item \textbf{Unidade amostral:}~
		
		\item \textbf{Tamanho da amostra (n):}~
		
		\item \textbf{Parâmetro de interesse $\left(\varTheta\right)$:}~
		
		\item \textbf{Técnica de amostragem:}~
		
		\item \textbf{Variável:}~
		
		\item \textbf{Unidade de medida da variável:}~
		
		\item \textbf{Classificação da variável:}~
		
		\item \textbf{Escala de medição da variável:}~
	\end{itemize}

	\subitem Uma pesquisadora pretende estimar o rendimento médio e o número médio de pessoas
das 18.540 famílias de uma cidade. Para isso, ela selecionou aleatoriamente 624
famílias, levando em conta o percentual de residências em cada bairro daquela cidade.
	
	\begin{itemize}	
		\item \textbf{População:}~
		
		\item \textbf{Unidade elementar:}~
		
		\item \textbf{Tamanho da população (N):}~
		
		\item \textbf{Amostra:}~
		
		\item \textbf{Unidade amostral:}~
		
		\item \textbf{Tamanho da amostra (n):}~
		
		\item \textbf{Parâmetro de interesse $\left(\varTheta\right)$:}~
		
		\item \textbf{Técnica de amostragem:}~
		
		\item \textbf{Variável:}~
		
		\item \textbf{Unidade de medida da variável:}~
		
		\item \textbf{Classificação da variável:}~
		
		\item \textbf{Escala de medição da variável:}~
	\end{itemize}

	\item Para avaliar uma localidade, o Índice de Desenvolvimento Humano (IDH) foi criado, no
final da década de 1990, com o objetivo de oferecer um contraponto a outro índice
muito usado até então, o Produto Interno Bruto (PIB), criado no final da década de 1930.
	Ambos os índices, PIB e IDH, foram reconhecidos internacionalmente e seus criadores
receberam o Prêmio Nobel de Economia, naquelas ocasiões. O cálculo do IDH está à
cargo do Programa das Nações Unidas para o Desenvolvimento (PNUD) e abrange
três dimensões, educação, saúde ou longevidade e renda. A imagem \ref{fig:imagem-da-tabela-1}, a seguir,
apresenta alguns dados sobre a Região Metropolitana de Curitiba. Analise essa imagem
e escreva algumas observações coerentes com os dados apresentados.

	\begin{figure}[h]
		\centering
		\includegraphics[width=\linewidth]{"fig/imagem da tabela 1"}
		\caption[Imagem da tabela 1]{Imagem da tabela 1 retirada da lista de exercícios 1}
		\label{fig:imagem-da-tabela-1}
	\end{figure}

	\textbf{Resolução:}
	
	\item Defina a variável "emoção" de modo que ela tenha:
	
	\subitem escala nominal: 
	
	\subitem escala de proporcionalidade (ou escala de razão):
	
	\item Defina a variável "corrente elétrica" de modo que ela tenha:
	
	\subitem escala nominal:
	
	\subitem escala ordinal:
	
	\subitem escala de proporcionalidade (ou escala de razão)
	
	\item Visite o sítio eletrônico do Programa das Nações Unidas para o Desenvolvimento
(PNUD),
em
<https://www.undp.org>,
e
o
do
PNUD
Brasil,
em
<http://www.br.undp.org>. No do PNUD Brasil, explore os links "IDH", "Atlas do
Desenvolvimento Humano" e o "Atlas do Desenvolvimento Humano no Brasil". Visite e
explore, ainda, o sítio eletrônico do Instituto Paranaense de Desenvolvimento
Econômico e Social (IPARDES), em <http://www.ipardes.pr.gov.br>, e o do Instituto
Brasileiro de Geografia e Estatística (IBGE), em <https://www.ibge.gov.br/>. No do
IBGE, explore o recurso eletrônico "IBGE Países", em <https://paises.ibge.gov.br/\#/>,
e o "IBGE Educa", em <https://educa.ibge.gov.br/>. Acesso em 16/03/2020.
	
	\item Identifique o tipo de amostragem usado em cada caso: se aleatória simples,
estratificada, sistemática ou de conveniência:
	
	\subitem Uma revista baseou suas conclusões sobre um escândalo político, nas respostas
	daqueles leitores que acessaram a página eletrônica da revista e emitiram suas
	opiniões a respeito de tal escândalo. 
	
	\textbf{Resolução:}~
	
	\subitem Uma professora escreveu o nome de cada um de seus 40 alunos, separadamente,
em um pedaço de papel; depois, ela misturou esses 40 papéis e extraiu 10 nomes.
	
	\textbf{Resolução:}~
	
	\subitem Um operário retira da esteira de produção uma peça a cada meia hora, para inspeção
e controle da qualidade. 
	
	\textbf{Resolução:}~
	
	\subitem Um programa sobre planejamento familiar pesquisou 800 homens e 800 mulheres
sobre seus pontos de vista a respeito do uso de pílulas anticoncepcionais e de
preservativos masculinos.
	
	\textbf{Resolução:}~
	
	\subitem Para o teste de uma vacina, um pesquisador utilizou-se de 250 voluntários.
	
	\textbf{Resolução:}~
	
	\item Um reflorestamento, no formato de um retângulo, tem 950 árvores de eucalipto, sendo
50 árvores plantadas em linha, no sentido do comprimento do terreno, e 19 árvores
plantadas em linha, no sentido da largura do terreno. Todas elas foram plantadas ao mesmo
tempo e com o mesmo espaçamento. No período de seu crescimento, todas elas
receberam luz, água, adubo, dentre outros, de forma semelhante. Deseja-se selecionar
uma amostra de 25 dessas árvores para fazer estudos inferenciais.
	
	\subitem Como você selecionará essa amostra usando amostragem aleatória simples?
	
	\subitem Como você selecionará essa amostra usando amostragem sistemática?
	
	\subitem Qual amostragem é mais indicada, de modo a respeitar os princípios da aleatoriedade,
da representatividade e da economicidade? Justifique sua resposta.
	
	\item Numa escola, há 1125 estudantes, sendo que 642 declaram-se do sexo masculino, 457
declaram-se do sexo feminino e 26 declaram-se de outros sexos. Um levantamento estatístico
será realizado e necessita de uma amostra probabilística de 120 estudantes, sendo suposto a
variável sexo influenciar nos resultados.
	
	\subitem Qual a técnica de amostragem indicada?
	
	
	\subitem Calcule o número de estudantes do sexo masculino, o número de estudantes do sexo
feminino e o número de estudantes de outros sexos que a amostra deve ter. Apresente o
resultado numa tabela, com a coluna da população e a coluna da amostra.
	
	\subitem Como você selecionará esse número de estudantes de cada sexo, para compor a
amostra, de modo a respeitar os princípios da aleatoriedade, representatividade e
economicidade?
	
	\item Uma instituição de ensino superior possui os seguintes cursos de graduação:
Engenharia Mecânica, Engenharia Eletrônica, Engenharia Civil, Arquitetura e Urbanismo,
Medicina, Odontologia, Tecnologia em Processos Ambientais, Tecnologia em Radiologia,
Licenciatura em Letras, Licenciatura em Física, Licenciatura em Matemática, Licenciatura
em História, Bacharelado em Estatística, Bacharelado em Química, Bacharelado em
Design, Bacharelado em Administração. Cada curso, alocado num prédio próprio, está
situado em locais diferentes de um grande campus universitário. Deseja-se fazer um
levantamento estatístico sobre o conhecimento dos estudantes dessa instituição, a respeito
da reciclagem do lixo doméstico. Presume-se que os estudantes dos diferentes cursos
tenham conhecimento similar sobre o assunto, com exceção dos estudantes de Tecnologia
em Processos Ambientais. Como você selecionará uma amostra probabilística dos
estudantes dessa instituição, de modo a respeitar os princípios da aleatoriedade, da
representatividade e da economicidade?
	
	
	\item Um pesquisador enviou, por e-mail, um questionário a um grupo de 5.000 egressos de
uma instituição de ensino superior e recebeu o questionário preenchido de 750 deles. Dos
750 respondentes, 63\% afirmaram que não conseguiram emprego na respectiva área de
	formação, no primeiro ano após a formatura. Esse pesquisador divulgou que “63\% dos
egressos daquela instituição de ensino levam mais de um ano após sua formatura, para
conseguir um emprego em sua área de formação”. Essa divulgação está coerente com a
pesquisa realizada? Justifique.
	
	
	\item Um levantamento de dados realizado numa universidade brasileira pretende obter e
analisar as alturas dos estudantes que respondem ao seguinte item: “Registre sua altura,
em polegadas”.
	
	\subitem Identifique dois problemas nesse levantamento.
	
	\subitem Como você agiria para levantar as alturas dos estudantes, de modo a coletar dados
fidedignos?
	
	\item Nos itens abaixo, justifique se a técnica de amostragem usada é tendenciosa ou não:
	
	\subitem Numa pesquisa sobre o salário médio dos funcionários de empresas de construção
civil, em certa empresa sorteada, a pesquisadora deparou-se com uma população
heterogênea em relação à variável de estudo: os oito diretores da empresa recebiam
salários entre R\$ 9.500,00 a R\$ 11.200,00; os 45 gerentes recebiam de R\$ 4.500,00
a R\$ 5.300,00; os 320 técnicos, de R\$ 2.400,00 a 3.850,00; e os 200 atendentes, de
R\$ 950,00 a R\$ 1.560,00. Como, nessa empresa, a amostra devia ser de 80
funcionários, a pesquisadora sorteou-os, aleatoriamente, pelo processo de
cumbuca, ou seja, fez uma amostragem aleatória simples, e calculou o salário médio
desses 80 funcionários.
	
	\subitem Para investigar a proporção de operários de uma fábrica favoráveis à mudança do
início das atividades, das 7h para às 7h30min, foi decidido entrevistar os 30 primeiros
operários que chegassem à fábrica, na quarta-feira.
	
	\subitem Para verificar o efeito do brinde nas vendas de sabão em pó de determinada marca,
foi constituída uma amostra com oito supermercados da zona sul e oito
supermercados da zona norte de uma metrópole. Nos oito supermercados da zona
sul, o produto foi vendido com brinde, enquanto nos outros oito foi vendido sem
brinde. No fim de um mês, comparou-se as vendas da zona sul e da zona norte.8
	
	\subitem Para determinar o número médio de pessoas que vivem numa casa, foi planejado
visitar 1.000 casas. Como o entrevistador não achou ninguém em 133 das 1.000
casas que devia visitar, ele visitou as casas vizinhas a essas 133, completando,
assim, a amostra de 1.000 casas.
	
	\subitem Para determinar o número médio de pessoas que vivem numa casa, foi planejado
visitar 1.000 casas. Como o entrevistador não achou ninguém em 133 das 1.000
casas que devia visitar, ele visitou as casas vizinhas a essas 133, completando,
assim, a amostra de 1.000 casas.
	
	\item Os 687 alunos do Ensino Médio, do período noturno, de uma escola estão distribuídos
em três séries: 1\textsuperscript{a}, 2\textsuperscript{a} e 3\textsuperscript{a}. Pretende-se fazer um estudo para conhecer os hábitos de fumar
desses alunos. Decidiu-se então fazer uma entrevista com 150 alunos. Para constituir uma
amostra foram feitas duas propostas. A primeira sugeria escrever os nomes de todos os
alunos em cédulas, misturar bem e tirar 150. A segunda sugeria pedir aos professores que
indicassem os nomes dos alunos que eles achavam que deveriam fazer parte da amostra.
Discuta essas duas propostas e, se for o caso, faça uma terceira.
	
	\item Assinale V se a sentença for verdadeira e F se for falsa. Justifique as falsas:
	
	
	\subitem (~~) Uma das finalidades principais da Estatística Aplicada é conhecer certas
	características de uma população ou universo, com base nos dados de uma amostra
	representativa desse universo.
	
	
	\subitem (~~) Se os objetivos de uma pesquisa não forem elaborados de forma bastante clara, há
um grande risco de ela ser inviabilizada a qualquer momento, pois todas as etapas
da pesquisa tomam esses objetivos como base.
	
	\subitem (~~) Na elaboração de um questionário para entrevistas é preferível fazer questões abertas,
ou seja, aquelas que podem ser respondidas livremente pelo entrevistado, não sendo,
portanto, oferecidas alternativas por parte do entrevistador; pois esse tipo de questão é
mais fácil de ser elaborada, bem como, mais fácil de serem apuradas as respostas.
	
	\subitem (~~) Pesquisa-piloto é uma pesquisa prévia que deve ser feita numa pequena amostra da
população em estudo, antes da coleta geral dos dados, a fim de detectar possíveis
falhas nos instrumentos de coleta de dados (formulários, questionários, aparelhos de
medição, por exemplo) ou no próprio processo dessa coleta.
		
	\subitem (~~) Os gráficos e tabelas são ferramentas estatísticas usadas para apresentar de forma
resumida, objetiva e clara os dados coletados e apurados.
	
	\subitem (~~) Um gráfico bonito e bem apresentado esteticamente é um gráfico confiável.
	
	\subitem (~~) Não se pode dar garantia de que a estimativa tenha valor igual ou mesmo um valor
próximo do parâmetro que se pretende estimar. No entanto, se a amostra for
suficientemente grande e obtida por meio de técnica de amostragem adequada, a
estimativa será próxima do parâmetro, na maioria das vezes. Ainda, amostras
sucessivas e criteriosas da mesma população tendem a fornecer estimativas
similares entre si e com valores em torno do parâmetro.
	
	\subitem (~~) Num estudo estatístico, tendência, viés ou vício é uma diferença sistemática entre a
estatística e o parâmetro que se quer estimar.
	
	\subitem (~~) Do ponto de vista científico, é preferível o estudo cuidadoso de uma amostra do que
o estudo rápido de toda a população que se deseja investigar.
	
	\subitem (~~) Em Estatística, a preocupação central é com o estudo de fenômenos coletivamente
típicos, ou seja, aqueles que apresentam regularidade na massa de observações, não
necessariamente nos casos isolados. Dessa forma, estudos de caso obtidos de
amostras muito pequenas não servem de base para se estimar o comportamento da
população de onde esses casos provieram, pois a probabilidade de obter uma
	estimativa que se desvia muito do parâmetro aumenta quando a amostra é pequena.
		
	\subitem (~~) Do ponto de vista científico, geralmente é mais relevante o tamanho absoluto da amostra
do que a porcentagem que ela representa da população, pois não é o tamanho, apenas,
que determina se a amostra é representativa da população de onde ela proveio.
	
	\subitem (~~) Na maioria das pesquisas estatísticas, a pessoa a ser entrevistada geralmente não
sabe o uso que se fará das informações que ela fornecerá. Essa prática está
eticamente correta, pois, afinal, as conclusões proporcionadas pela pesquisa
importam, sobretudo, para aqueles que pagaram para obter as informações.
	
	\subitem (~~) Nas pesquisas com seres humanos, como os testes de medicamentos, é permitido
que o pesquisador investigue suas hipóteses científicas em grupos de pessoas que
não saibam que estão participando de uma pesquisa. Isso porque a investigação
somente a partir de voluntários não garantiria ao pesquisador conclusões confiáveis
para fazer inferências, isto é, estender as conclusões retiradas com base na amostra
para toda a população de onde essa amostra proveio.
	
\end{enumerate}



