%% abtex2-modelo-trabalho-academico.tex, v-1.9.2 laurocesar
%% Copyright 2012-2017 by abnTeX2 group at http://abntex2.googlecode.com/ 
%%
%% This work may be distributed and/or modified under the
%% conditions of the LaTeX Project Public License, either version 1.3
%% of this license or (at your option) any later version.
%% The latest version of this license is in
%%   http://www.latex-project.org/lppl.txt
%% and version 1.3 or later is part of all distributions of LaTeX
%% version 2005/12/01 or later.
%%
%% This work has the LPPL maintenance status `maintained'.
%% 
%% The Current Maintainer of this work is Emílio Eiji Kavamura,
%% eek.edu@outlook.com; emilio.kavamura@ufpr.br
%% Further information about abnTeX2 are available on 
%%
%% http://abntex2.googlecode.com/
%%
%% https://code.google.com/p/abntex2/issues/ 
%%
%% Further information about UFPR abnTeX2 are available on 
%%
%% https://github.com/eekBR/ufpr-abntex/
%%
%% This work consists of the files 
% 
%          main.tex   programa principal
%      00-dados.tex   entrada de dados 
%    00-pacotes.tex   pacotes carregados no modelo
% 00-pretextual.tex   processamento dos elementos pre-textuais
%          UFPR.sty   ajusta do modelo canonico às normas  UFPR
%
%    referencias.bib
%                     e outras arquivos de imagens
%
%
%------------------------------------------------------------------------
% ------------------------------------------------------------------------
% abnTeX2: Modelo de Trabalho Academico (tese de doutorado, dissertacao de
% mestrado e trabalhos monograficos em geral) em conformidade com 
% ABNT NBR 6023:2018: Informação e documentação - Referências - Elaboração
% ------------------------------------------------------------------------
% ------------------------------------------------------------------------
%
% DATA DE ATUALIZAÇÃO: 2020-06-10

\documentclass[
        % -- opções da classe memoir --
        12pt,                           % tamanho da fonte
        openright,                      % capítulos começam em pág ímpar (insere página vazia caso preciso)
        %twoside,                        % para impressão em verso e anverso. Oposto a oneside
        oneside,
        a4paper,                        % tamanho do papel. 
        % -- opções da classe abntex2 --
        chapter=TITLE,         % títulos de capítulos convertidos em letras maiúsculas
        section=TITLE,         % títulos de seções convertidos em letras maiúsculas
        subsection=Title,      % títulos de subseções convertidos em letras maiúsculas
        %subsubsection=TITLE,  % títulos de subsubseções convertidos em letras maiúsculas
        % -- opções do pacote babel --
        english,                        % idioma adicional para hifenização
        %french,                         % idioma adicional para hifenização
        spanish,                        % idioma adicional para hifenização
        portugues,                      % o último idioma é o principal do documento
        %%%%%%%%%%%%
        %eek: colocação da opção para o sumario ter formatação tradicional
        sumario=tradicional             % título no formato tradicional
        ]{abntex2}


\usepackage{UFPR}
% Pacotes básicos 
% ----------------------------------------------------------
%\usepackage{lmodern}			% Usa a fonte Latin Modern			
\usepackage[T1]{fontenc}		% Selecao de codigos de fonte.
\usepackage[utf8]{inputenc}		% Codificacao do documento (conversão automática dos acentos)
\usepackage{lastpage}			% Usado pela Ficha catalográfica
\usepackage{indentfirst}		% Indenta o primeiro parágrafo de cada seção.
\usepackage{color}		    	% Controle das cores
\usepackage{graphicx}			% Inclusão de gráficos
\usepackage{microtype} 			% para melhorias de justificação
\usepackage{ifthen}		    	% para montar condicionais
\usepackage[brazil]{babel}		% para utilizar termos em portugues
\usepackage[final]{pdfpages}    % para incluir páginas de arquivos pdf
\usepackage{lipsum}				% para geração de dummy text
\usepackage{csquotes}

%\usepackage[style=long]{glossaries}
%\usepackage{abntex2glossaries}


\usepackage{cancel} 		% permite representar o cancelamento de termos em texto ou equacoes	
\usepackage{xcolor} 		% cores extendidas	
\usepackage{smartdiagram}   	% gera diagramas a partir de listas
%\usepackage{float} 		% Para a figura ficar na posição correta	    
\usepackage{textcomp} 		% supporte para fontes da Text Companion 
\usepackage{longtable}		% uso de longtable
\usepackage{amsmath}		% simbolos matematicos
\usepackage{lscape}		% páginas em paisagem
\usepackage{multicol}		% mescla de colunas em tabelas
\usepackage{multirow}		% mescla de linhas em tabelas
\usepackage{newfloat} 		% criação do indice de quadros
%\usepackage{caption} 		% configura legenda 
	%[format=plain]
	%\renewcommand\caption[1]{%
    	%\captionsetup{font=small}	% tamanho da fonte 10pt
    	%,format=hang
 	% \caption{#1}}
	%\captionsetup{width=0.8\textwidth}
\captiondelim{-- }
\captiontitlefont{\small}
\captionnamefont{\small}

% Pacotes de citações BibLaTeX
% ----------------------------------------------------------
\usepackage[style=abnt,
	backref=true,
	backend=biber,
	citecounter=true,
	backrefstyle=three, 
	url=true,
	maxbibnames=99,
    mincitenames=1,
    maxcitenames=2,
    backref=true,
    hyperref=true,
    firstinits=true,
    uniquename=false,
    uniquelist=false]{biblatex}

% Texto padrão para as referências
% ----------------------------------------------------------
\DefineBibliographyStrings{brazil}{%
	 backrefpage  = {Citado \arabic{citecounter} vez na página},		% originally "cited on page"
	 backrefpages = {Citado \arabic{citecounter} vezes nas páginas},	% originally "cited on pages"
	 urlfrom      = {Dispon\'ivel em},
}

% Ajusta indentação de Referencias no ToC
% ----------------------------------------------------------
\defbibheading{bay}[\bibname]{%
  \chapter*{#1}%
  \markboth{#1}{#1}%
  \addcontentsline{toc}{chapter}
  {\protect\numberline{}\bibname}
}

% Formatando o avançao dos títulos no sumário 
% ----------------------------------------------------------
\makeatletter
	\pretocmd{\chapter}{\addtocontents{toc}{\protect\addvspace{-12\p@}}}{}{}
	\pretocmd{\section}{\addtocontents{toc}{\protect\addvspace{-3\p@}}}{}{}
\makeatother

% Para retirar os símbolos <> da URL  
% ----------------------------------------------------------
\DeclareFieldFormat{illustrated}{\addspace #1\isdot}%
%\DeclareFieldFormat{url}{\bibstring{urlform}\addcolon\addspace<\url{#1}>}%
%\DeclareFieldFormat{url}{\bibstring{urlfrom}\addcolon\addspace<\url{#1}>}%
\DeclareFieldFormat{url}{\bibstring{urlfrom}\addcolon \space\addspace{#1}} 
% remove <> em urls de acordo com abnt-6023:2018	

% Ajustar o espaço para a formatação da data
% ----------------------------------------------------------
\DeclareFieldFormat{urldate}{\bibstring{urlseen}\addcolon\addspace #1}%
\DeclareFieldFormat*{note}{\addspace #1}%

% Para ajustar o tamanho da fonte do número da primeira página do capítulo
% comando utilizado na parte textual 
% ----------------------------------------------------------
\makepagestyle{chapfirst}% Just for the first page of a chapter
\makeoddhead{chapfirst}{}{}{\footnotesize{\thepage}}

%%criar um novo estilo de cabeçalhos e rodapés
\makepagestyle{simplestextual}
  %%cabeçalhos
  \makeevenhead{simplestextual} %%pagina par
     {}{}{\footnotesize \thepage}
     
  \makeoddhead{simplestextual} %%pagina ímpar ou com oneside
     {}{}{\footnotesize \thepage}
  %\makeheadrule{simplestextual}{\textwidth}{\normalrulethickness} %linha
  %% rodapé
  \makeevenfoot{simplestextual}
     {}{}{} %%pagina par
      
  \makeoddfoot{simplestextual} %%pagina ímpar ou com oneside
     {}{}{}
     
% Define a formatação dos capítulos póstextuais numerados
% ----------------------------------------------------------
\newcommand{\poschap}[1]{
	\stepcounter{chapter}
	\markboth{#1}{#1}%
	\pdfbookmark[2]{#1}{#1}
	\addtocontents{toc}{\vspace{-0pt}}
	\addcontentsline{toc}{chapter}{\hspace{14.5mm}\textbf{\appendixname~
	\thechapter~- #1}}
	\chapter*{\appendixname\space\space\thechapter~- \uppercase{#1}}%
	{}
}
\newcommand{\refap}[1]{\hyperref[#1]{Apêndice~\ref{#1}}} 	% Referência apÊndices
% uso do tikz e pgfplots
% ----------------------------------------------------------
%\usetikzlibrary{external}
\usetikzlibrary{arrows,calc,patterns,angles,quotes}
\usepackage{pgfplots}
\pgfplotsset{compat=1.15}

% Define o comando para citação de fontes em elementos gráficos (figuras, imagens,...).
% ----------------------------------------------------------
%  AUTOR(ano)
%
% parâmetro é a bibkey da fonte
  
\newcommand{\citefg}[1]{~\citeauthor{#1}(\citeyear{#1})}


%%%%%%%%%%%%%%%%%%%%%%%%%%%%%%%%%%%%%%%%%%%%%%%%%%%%%%%
% Arquivo para entrada de dados para a parte pré textual
%%%%%%%%%%%%%%%%%%%%%%%%%%%%%%%%%%%%%%%%%%%%%%%%%%%%%%%
% 
% Basta digitar as informações indicidas, no formato 
% apresentado.
%
%%%%%%%
% Os dados solicitados são, na ordem:
%
% tipo do trabalho
% componentes do trabalho 
% título do trabalho
% nome do autor
% local 
% data (ano com 4 dígitos)
% orientador(a)
% coorientador(a)(as)(es)
% arquivo com dados bibliográficos
% instituição
% setor
% programa de pós gradução
% curso
% preambulo
% data defesa
% CDU
% errata
% assinaturas - termo de aprovação
% resumos & palavras chave
% agradecimentos
% dedicatoria
% epígrafe


% Informações de dados para CAPA e FOLHA DE ROSTO
%----------------------------------------------------------------------------- 
\tipotrabalho{Trabalho Acadêmico}
%    {Relatório Técnico}
%    {Dissertação}
%    {Tese}
%    {Monografia}

% Marcar Sim para as partes que irão compor o documento pdf
%----------------------------------------------------------------------------- 
 \providecommand{\terCapa}{Sim}
 \providecommand{\terFolhaRosto}{Sim}
 \providecommand{\terTermoAprovacao}{Nao}
 \providecommand{\terDedicatoria}{Nao}
 \providecommand{\terFichaCatalografica}{Nao}
 \providecommand{\terEpigrafe}{Nao}
 \providecommand{\terAgradecimentos}{Nao}
 \providecommand{\terErrata}{Nao}
 \providecommand{\terListaFiguras}{Sim}
 \providecommand{\terListaQuadros}{Sim}
 \providecommand{\terListaTabelas}{Sim}
 \providecommand{\terSiglasAbrev}{Nao}
 \providecommand{\terResumos}{Nao}
 \providecommand{\terSumario}{Sim}
 \providecommand{\terAnexo}{Nao}
 \providecommand{\terApendice}{Nao}
 \providecommand{\terIndiceR}{Nao}
%----------------------------------------------------------------------------- 

\titulo{Trabalho acadêmico de probabilidade e estatística, turma }
\autor{Erika Angela}
\local{Curitiba}
\data{2020} %Apenas ano 4 dígitos

% Orientador ou Orientadora
\orientador{}
%Prof Emílio Eiji Kavamura, MSc}
\orientadora{
Prof\textordfeminine~Grace Kelly, DSc}
% Pode haver apenas uma orientadora ou um orientador
% Se houver os dois prevalece o feminino.

% Em termos de coorientação, podem haver até quatro neste modelo
% Sendo 2 mulhere e 2 homens.
% Coorientador ou Coorientadora
\coorientador{}%Prof Morgan Freeman, DSc}
\coorientadora{Prof\textordfeminine~Audrey Hepburn, DEng}

% Segundo Coorientador ou Segunda Coorientadora
\scoorientador{}
%Prof Jack Nicholson, DEng}
\scoorientadora{}
%Prof\textordfeminine~Ingrid Bergman, DEng}
% ----------------------------------------------------------
\addbibresource{referencias.bib}

% ----------------------------------------------------------
\instituicao{%
Universidade Federal do Paraná}

\def \ImprimirSetor{}%
%Setor de Tecnologia}

\def \ImprimirProgramaPos{}%Programa de Pós Graduação em Engenharia de Construção Civil}

\def \ImprimirCurso{}%
%Curso de Engenharia Civil}

\preambulo{
Trabalho apresentado como requisito parcial para a obtenção do título de Mestre em Ciências pelo Programa de Pós Graduação em Engenharia de Construção Civil do  Setor de Tecnologia  da Universidade Federal do Paraná}
%do grau de Bacharel em Expressão Gráfica no curso de Expressão Gráfica, Setor de Exatas da Universidade Federal do Paraná}

%----------------------------------------------------------------------------- 

\newcommand{\imprimirCurso}{}
%Programa de P\'os Gradua\c{c}\~ao em Engenharia da Constru\c{c}\~ao Civil}

\newcommand{\imprimirDataDefesa}{
09 de Dezembro de 2018}

\newcommand{\imprimircdu}{
02:141:005.7}

% ----------------------------------------------------------
\newcommand{\imprimirerrata}{
Elemento opcional da \cites[4.2.1.2]{NBR14724:2011}. Exemplo:

\vspace{\onelineskip}

FERRIGNO, C. R. A. \textbf{Tratamento de neoplasias ósseas apendiculares com
reimplantação de enxerto ósseo autólogo autoclavado associado ao plasma
rico em plaquetas}: estudo crítico na cirurgia de preservação de membro em
cães. 2011. 128 f. Tese (Livre-Docência) - Faculdade de Medicina Veterinária e
Zootecnia, Universidade de São Paulo, São Paulo, 2011.

\begin{table}[htb]
\center
\footnotesize
\begin{tabular}{|p{1.4cm}|p{1cm}|p{3cm}|p{3cm}|}
  \hline
   \textbf{Folha} & \textbf{Linha}  & \textbf{Onde se lê}  & \textbf{Leia-se}  \\
    \hline
    1 & 10 & auto-conclavo & autoconclavo\\
   \hline
\end{tabular}
\end{table}}

% Comandos de dados - Data da apresentação
\providecommand{\imprimirdataapresentacaoRotulo}{}
\providecommand{\imprimirdataapresentacao}{}
\newcommand{\dataapresentacao}[2][\dataapresentacaoname]{\renewcommand{\dataapresentacao}{#2}}

% Comandos de dados - Nome do Curso
\providecommand{\imprimirnomedocursoRotulo}{}
\providecommand{\imprimirnomedocurso}{}
\newcommand{\nomedocurso}[2][\nomedocursoname]
  {\renewcommand{\imprimirnomedocursoRotulo}{#1}
\renewcommand{\imprimirnomedocurso}{#2}}


% ----------------------------------------------------------
\newcommand{\AssinaAprovacao}{

\assinatura{%\textbf
   {Professora} \\ UFPR}
   \assinatura{%\textbf
   {Professora} \\ ENSEADE}
   \assinatura{%\textbf
   {Professora} \\ TIT}
   %\assinatura{%\textbf{Professor} \\ Convidado 4}
      
   \begin{center}
    \vspace*{0.5cm}
    %{\large\imprimirlocal}
    %\par
    %{\large\imprimirdata}
    \imprimirlocal, \imprimirDataDefesa.
    \vspace*{1cm}
  \end{center}
  }
  
% ----------------------------------------------------------
%\newcommand{\Errata}{%\color{blue}
%Elemento opcional da \textcite[4.2.1.2]{NBR14724:2011}. Exemplo:
%}

% ----------------------------------------------------------
\newcommand{\EpigrafeTexto}{%\color{blue}
\textit{``Não vos amoldeis às estruturas deste mundo, \\
		mas transformai-vos pela renovação da mente, \\
		a fim de distinguir qual é a vontade de Deus: \\
		o que é bom, o que Lhe é agradável, o que é perfeito.\\
		(Bíblia Sagrada, Romanos 12, 2)}
}

% ----------------------------------------------------------
\newcommand{\ResumoTexto}{%\color{blue}
Segundo a \textcite[3.1-3.2]{abntex2modelo}, o resumo deve ressaltar o  objetivo, o método, os resultados e as conclusões do documento. A ordem e a extensão destes itens dependem do tipo de resumo (informativo ou indicativo) e do tratamento que cada item recebe no documento original. O resumo deve ser precedido da referência do documento, com exceção do resumo inserido no próprio documento. (\ldots) As palavras-chave devem figurar logo abaixo do  resumo, antecedidas da expressão Palavras-chave:, separadas entre si por ponto e finalizadas também por ponto.
}

\newcommand{\PalavraschaveTexto}{%\color{blue}
latex. abntex. editoração de texto.}

% ----------------------------------------------------------
\newcommand{\AbstractTexto}{%\color{blue}
This is the english abstract.
}
% ---
\newcommand{\KeywordsTexto}{%\color{blue}
latex. abntex. text editoration.
}

% ----------------------------------------------------------
\newcommand{\Resume}
{%\color{blue}
%Il s'agit d'un résumé en français.
} 
% ---
\newcommand{\Motscles}
{%\color{blue}
 %latex. abntex. publication de textes.
}

% ----------------------------------------------------------
\newcommand{\Resumen}
{%\color{blue}
%Este es el resumen en español.
}
% ---
\newcommand{\Palabrasclave}
{%\color{blue}
%latex. abntex. publicación de textos.
}

% ----------------------------------------------------------
\newcommand{\AgradecimentosTexto}{%\color{blue}
Os agradecimentos principais são direcionados à Gerald Weber, Miguel Frasson, Leslie H. Watter, Bruno Parente Lima, Flávio de  Vasconcellos Corrêa, Otavio Real Salvador, Renato Machnievscz\footnote{Os nomes dos integrantes do primeiro
projeto abn\TeX\ foram extraídos de \url{http://codigolivre.org.br/projects/abntex/}} e todos aqueles que contribuíram para que a produção de trabalhos acadêmicos conforme as normas ABNT com \LaTeX\ fosse possível.

Agradecimentos especiais são direcionados ao Centro de Pesquisa em Arquitetura da Informação\footnote{\url{http://www.cpai.unb.br/}} da Universidade de Brasília (CPAI), ao grupo de usuários
\emph{latex-br}\footnote{\url{http://groups.google.com/group/latex-br}} e aos novos voluntários do grupo \emph{\abnTeX}\footnote{\url{http://groups.google.com/group/abntex2} e
\url{http://abntex2.googlecode.com/}}~que contribuíram e que ainda
contribuirão para a evolução do \abnTeX.

Os agradecimentos principais são direcionados à Gerald Weber, Miguel Frasson, Leslie H. Watter, Bruno Parente Lima, Flávio de Vasconcellos Corrêa, Otavio Real Salvador, Renato Machnievscz\footnote{Os nomes dos integrantes do primeiro
projeto abn\TeX\ foram extraídos de \url{http://codigolivre.org.br/projects/abntex/}} e todos aqueles que contribuíram para que a produção de trabalhos acadêmicos conforme as normas ABNT com \LaTeX\ fosse possível.
}

% ----------------------------------------------------------
\newcommand{\DedicatoriaTexto}{%\color{blue}
\textit{ Este trabalho é dedicado às crianças adultas que,\\
   quando pequenas, sonharam em se tornar cientistas.}
	}



% compila o indice
% ----------------------------------------------------------

\makeindex
% ----------------------------------------------------------
% Início do documento
% ----------------------------------------------------------
\begin{document}
% ----------------------------------------------------------
% Adequando o uppercase titulo dos elementos nas suas respectivas legendas
% Definicoes que n\~ao funcionaram quando colocados no arquivo de estilos ou de pacotes

\renewcommand{\bibname}{{REFER\^ENCIAS}}
\renewcommand{\tablename}{TABELA }
\renewcommand{\figurename}{FIGURA }
\renewcommand{\figureautorefname}{FIGURA}
\renewcommand{\tableautorefname}{TABELA}
\renewcommand{\equationname}{EQUA\c{C}\~AO~}
\renewcommand{\equationautorefname}{EQUA\c{C}\~AO~}

% Para ajustar o tamanho da fonte do número da primeira página do capítulo
\aliaspagestyle{chapter}{chapfirst}% customizing chapter pagestyle

% ELEMENTOS PRÉ-TEXTUAIS
\makeoddhead{chapfirst}{}{}{}
% ----------------------------------------------------------
% Capa
% ----------------------------------------------------------
 \ifthenelse{\equal{\terCapa}{Sim}}{
\imprimircapa}{}

% Folha de rosto
% ----------------------------------------------------------
\imprimirfolhaderosto*

% Inserir a ficha bibliografica
% ----------------------------------------------------------
 \ifthenelse{\equal{\terFichaCatalografica}{Sim}}
 {\insereFichaCatalografica{}\cleardoublepage}
 {}

% Inserir errata
% ----------------------------------------------------------
 \ifthenelse{\equal{\terErrata}{Sim}}
 {\begin{errata}%\color{blue}
   \imprimirerrata
  \end{errata}}
 {}

% Inserir folha de aprovação
% ----------------------------------------------------------
\ifthenelse{\equal{\terTermoAprovacao}{Sim}}{
\insereAprovacao}{}

% Dedicatória
% ----------------------------------------------------------
\ifthenelse{\equal{\terDedicatoria}{Sim}}{
\begin{dedicatoria}
   \vspace*{\fill}
   \centering
   \noindent
   \DedicatoriaTexto
   \vspace*{\fill}
\end{dedicatoria}
}{}

% Agradecimentos
% ----------------------------------------------------------

 \ifthenelse{\equal{\terAgradecimentos}{Sim}}
 {\begin{agradecimentos}
    \AgradecimentosTexto
  \end{agradecimentos}
  }{}
% Epígrafe
% ----------------------------------------------------------

\ifthenelse{\equal{\terEpigrafe}{Sim}}{
\begin{epigrafe}
    \vspace*{\fill}
	\begin{flushright}
        \EpigrafeTexto
	\end{flushright}
\end{epigrafe}
}{}

% RESUMOS
% ----------------------------------------------------------
% resumo em português
%\setlength{\absparsep}{18pt} % ajusta o espaçamento dos parágrafos do resumo
 \ifthenelse{\equal{\terResumos}{Sim}}{
\begin{resumo}
    \ResumoTexto
    
    %\vspace{\onelineskip}
    \noindent 
    \textbf{Palavras-chaves}: \PalavraschaveTexto
\end{resumo}

%% resumo em inglês
\begin{resumo}[ABSTRACT]
 \begin{otherlanguage*}{english}
   \AbstractTexto
   
   %\vspace{\onelineskip}
   \noindent 
   \textbf{Key-words}: \KeywordsTexto
 \end{otherlanguage*}
\end{resumo}


% resumo em francês 
\ifthenelse{\equal{\Resume}{}}
{}
{
 \begin{resumo}[RESUME]%Résumé
  \begin{otherlanguage*}{french}
     \Resume
     
     %\vspace{\onelineskip}
     \noindent      
     \textbf{Mots clés}: \Motscles
  \end{otherlanguage*}
 \end{resumo}
} 

% resumo em espanhol
\ifthenelse{\equal{\Resume}{}}{}
{ \begin{resumo}[RESUMEN]
  \begin{otherlanguage*}{spanish}
    \Resumen 
   
   %\vspace{\onelineskip}
   \noindent    
    \textbf{Palabras clave}: \Palabrasclave
  \end{otherlanguage*}
 \end{resumo}
}
}{}

% inserir lista de ilustrações
% ----------------------------------------------------------
\ifthenelse{\equal{\terListaFiguras}{Sim}}{
%\pdfbookmark[0]{\listfigurename}{lof}
\listoffigures*
\cleardoublepage
}{}

% inserir lista de quadros
% ----------------------------------------------------------
\ifthenelse{\equal{\terListaQuadros}{Sim}}{
%\pdfbookmark[0]{\listtablename}{lot}
\listofquadros*
\cleardoublepage
}{}

% inserir lista de tabelas
% ----------------------------------------------------------
\ifthenelse{\equal{\terListaTabelas}{Sim}}{
%\pdfbookmark[0]{\listtablename}{lot}
\listoftables*
\cleardoublepage
}{}


% inserir lista de abreviaturas e siglas 
% inserir lista de símbolos
% ----------------------------------------------------------

 \ifthenelse{\equal{\terSiglasAbrev}{Sim}}{
    \imprimirlistadesiglas
    \cleardoublepage
    \imprimirlistadesimbolos
    \cleardoublepage
 }{}

% inserir o sumario
\makeoddhead{chapfirst}{}{}{}
% ----------------------------------------------------------
\ifthenelse{\equal{\terSumario}{Sim}}{
%\pdfbookmark[0]{\contentsname}{toc}
\tableofcontents*
%\cleardoublepage
}{}
 

 
 


% ----------------------------------------------------------
% ELEMENTOS TEXTUAIS
% ----------------------------------------------------------
\textual % \pagestyle{textualUFPR}

\pagestyle{simplestextual}
% sugerido por Youssef Cherem 20170316
% https://mail.google.com/mail/u/0/?tab=wm#inbox/15ad3fe6f4e5ff1f

% Introdução (exemplo de capítulo sem numeração, mas presente no Sumário)
% ----------------------------------------------------------
%\chapter[INTRODUÇÃO]{INTRODUÇÃO}
%---------------------------------------------------------------------------------------


\cite{ISO5122:1979}

\textcite{ISO5122:1979}

\begin{itemize}
 \item para inserir uma sigla: 
    \verb|\criarsigla|\{ABNT\}{Associa\c{c}\~ao Brasileira de Normas T\'ecnicas}:

    \criarsigla{ABNT}{Associa\c{c}\~ao Brasileira de Normas T\'ecnicas}

\item para inserir s\'imbolo: 
    \verb|\criarsimbolo{$ \Gamma $}{Letra grega Gama}|
  
    $ \Gamma $ \criarsimbolo{$ \Gamma $}{Letra grega Gama}

\end{itemize}

% Usado para testar o formato uppercase dos t\'itulos 
% em maiusculas nas respectivas listas
%---------------------------------------------------------------------------------------
\begin{table}[!ht]
 \centering
 \par\caption{TesTANDO TABELAS}

\begin{tabular}{c|c|c}
 teste1&teste1&teste1\\\hline\hline
  1&2&3\\\hline
 \end{tabular}
 \label{tab:tab01}
\end{table}



\tabela{tabela teste} %1 Título da tabela
{
\begin{tabular}{c|c|c|c|c|c}
 teste1&teste2&teste3&teste1&teste2&teste3\\\hline\hline
  1&2&3&4&5&6\\\hline
 \end{tabular}
} %2 Tabela
{\textcite{ISO5122:1979}}%3 Fonte da tabela
{teste1} %4 Label da tabela tab:teste1
{ Nota de teste } %5 Nota da tabela
{testando as figuras e tabelas, fda} %6 Legenda da tabela


\figura
{TESTE DE FIGURAS 2} %1 Legenda
{.55} %2  % da largura da área de texto
{fig/figure} %3 localização da figura
{\textcite[1]{abntex2modelo}} %4 fonte da figura
{teste} %5 etiqueta
{\url{https://goo.gl/EKFRak} TESTE DE FIGURAS 2 TESTE DE FIGURAS 2 TESTE DE FIGURAS 2 TESTE DE FIGURAS 2 TESTE DE FIGURAS 2 TESTE DE FIGURAS 2 TESTE DE FIGURAS 2 TESTE DE FIGURAS 2 TESTE DE FIGURAS 2 TESTE DE FIGURAS 2 TESTE DE FIGURAS 2} %6 Nota da figura
{} %7 Legenda da figura

\figura
{TESTE DE FIGURAS 3} % Legenda
{.65} % % da largura da área de texto
{fig/tipog} % localização da figura
{o Autor (2017)} % fonte da figura
{tipo1} % etiqueta
{}
{}

\figura
{Figura original} % Legenda
{.35} % % da largura da área de texto
{fig/fig} % localização da figura
{\textcite{luminaria01}} % fonte da figura
{tipo2} % etiqueta
{}
{}

\figurac
{Figura aparada} % Legenda
{.35} % % da largura da área de texto
{fig/fig} % localização da figura
{\textcite{luminaria01}} % fonte da figura
{tipo3} % etiqueta =>  fig:tipo3
{notinha}   % Nota
{legendonha} % legenda
{70} % laterais mm
{80} % superior e inferior mm


A \autoref{fig:teste} apresenta o seguinte detalhe que deve ser observado em pacientes com Wordnite

%---------------------------------------------------------------------------------------
\chapter{REVISÃO TEÓRICA}

\textcite[3.1-3.2]{abntex2modelo}

\textcite{luminaria01}

\textcite{luminaria01}


\textcite{luminaria01}



\textcite{luminaria01}

\textcite{ISO5122:1979}

\section{Modelo}

\subsection{Modelo simplificado}
\newpage
\textcite[3.1-3.2]{abntex2modelo}

\subsubsection{Modelo simplificadíssimo}


\chapter{Lista de Exercícios 1}

\begin{enumerate}[label=\emph{\arabic*})]
	\item Nos exercícios abaixo, identifique a população, a unidade elementar, o tamanho da	população, a amostra, a unidade amostral, o tamanho da amostra, o parâmetro de interesse, a técnica de amostragem, a variável de interesse e respectiva unidade de medida, classifique a variável e identifique a escala de medição da variável:

	      \begin{enumerate}[label=\emph{\alph*})]

		      \item A gerência de uma loja de eletrônicos deseja verificar se os consumidores que compraram uma televisão de 32 polegadas, nos últimos seis meses, estavam satisfeitos com o produto. A gerência planeja investigar 500 desses consumidores.

		            \begin{itemize}
			            \item \textbf{População:}~Todos os consumidores da loja que compraram uma televisão de 32 polegadas, nos últimos seis meses

			            \item \textbf{Unidade elementar:}~Cada consumidor da loja que comprou uma televisão de 32 polegadas, nos
			                  últimos seis meses

			            \item \textbf{Tamanho da população (N):}~não informado

			            \item \textbf{Amostra:}~Todos os consumidores da loja investigados pela gerência

			            \item \textbf{Unidade amostral:}~Cada consumidor da loja investigado pela gerência

			            \item \textbf{Tamanho da amostra (n):}~500 consumidores

			            \item \textbf{Parâmetro de interesse $\left(\varTheta\right)$:}~quantidade de consumidores satisfeitos com o produto; percentual de
			                  consumidores satisfeitos com o produto; proporção de consumidores satisfeitos com o produto

			            \item \textbf{Técnica de amostragem:}~não informado

			            \item \textbf{Variável:}~Satisfação com o produto (sim, não); Grau de satisfação com o produto (muito satisfeito,
			                  satisfeito, pouco satisfeito)

			            \item \textbf{Unidade de medida da variável:}~não há

			            \item \textbf{Classificação da variável:}~nominal, se a variável for "satisfação com o produto"; ordinal, se a variável
			                  for "grau de satisfação com o produto"

			            \item \textbf{Escala de medição da variável:}~escala nominal, se a variável "satisfação com o produto" tiver resultados
			                  "sim" ou "não"; escala ordinal, se a variável "grau de satisfação com o produto" tiver como resultados
			                  "muito satisfeito", "satisfeito", "pouco satisfeito"; escala de proporcionalidade (ou escala de razão) se a
			                  variável "grau de satisfação com o produto" for um número de 0 a 100, sendo "0" nada satisfeito e "100"
			                  totalmente satisfeito.
		            \end{itemize}

		      \item Numa pesquisa de intenção de voto para tentar prever o resultado de uma eleição para
		            prefeito, de 32.873 eleitores foram entrevistados 2.560, em 600 domicílios, respeitando-se a proporção de sexo, classe social, idade e bairro da residência dos eleitores.

		            \begin{itemize}
			            \item \textbf{População:}~Eleitores

			            \item \textbf{Unidade elementar:}~1 eleitor

			            \item \textbf{Tamanho da população (N):}~$32.873$ eleitores

			            \item \textbf{Amostra:}~Eleitores entrevistados

			            \item \textbf{Unidade amostral:}~Cada eleitor entrevistado

			            \item \textbf{Tamanho da amostra (n):}~$2.560$ eleitores em 600 domicílios

			            \item \textbf{Parâmetro de interesse $\left(\varTheta\right)$:}~intenção de voto de cada eleitor

			            \item \textbf{Técnica de amostragem:}~amostragem proporcional estratificada

			            \item \textbf{Variável:}~intenção de voto

			            \item \textbf{Unidade de medida da variável:}~voto

			            \item \textbf{Classificação da variável:}~variável quantitativa discreta

			            \item \textbf{Escala de medição da variável:}~a escala da variável será ordinal
		            \end{itemize}

		      \item Um lote contém $2.342$ peças de um mesmo tipo, não ordenadas. Retira-se dali, ao
		            acaso e sem reposição, 85 peças para estimar o percentual de peças defeituosas no
		            lote.

		            \begin{itemize}
			            \item \textbf{População:}~

			            \item \textbf{Unidade elementar:}~

			            \item \textbf{Tamanho da população (N):}~

			            \item \textbf{Amostra:}~

			            \item \textbf{Unidade amostral:}~

			            \item \textbf{Tamanho da amostra (n):}~

			            \item \textbf{Parâmetro de interesse $\left(\varTheta\right)$:}~

			            \item \textbf{Técnica de amostragem:}~

			            \item \textbf{Variável:}~

			            \item \textbf{Unidade de medida da variável:}~

			            \item \textbf{Classificação da variável:}~

			            \item \textbf{Escala de medição da variável:}~
		            \end{itemize}

		      \item A polícia rodoviária de uma localidade deseja estimar a proporção de carros com pneus
		            carecas que passam pelo posto policial, durante um mês. Assim, adota a prática de
		            parar um carro para inspeção dos pneus, a cada 30 minutos, durante duas semanas.

		            \begin{itemize}
			            \item \textbf{População:}~

			            \item \textbf{Unidade elementar:}~

			            \item \textbf{Tamanho da população (N):}~

			            \item \textbf{Amostra:}~

			            \item \textbf{Unidade amostral:}~

			            \item \textbf{Tamanho da amostra (n):}~

			            \item \textbf{Parâmetro de interesse $\left(\varTheta\right)$:}~

			            \item \textbf{Técnica de amostragem:}~

			            \item \textbf{Variável:}~

			            \item \textbf{Unidade de medida da variável:}~

			            \item \textbf{Classificação da variável:}~

			            \item \textbf{Escala de medição da variável:}~
		            \end{itemize}

		      \item Uma pesquisadora pretende estimar o rendimento médio e o número médio de pessoas
		            das 18.540 famílias de uma cidade. Para isso, ela selecionou aleatoriamente 624
		            famílias, levando em conta o percentual de residências em cada bairro daquela cidade.

		            \begin{itemize}
			            \item \textbf{População:}~

			            \item \textbf{Unidade elementar:}~

			            \item \textbf{Tamanho da população (N):}~

			            \item \textbf{Amostra:}~

			            \item \textbf{Unidade amostral:}~

			            \item \textbf{Tamanho da amostra (n):}~

			            \item \textbf{Parâmetro de interesse $\left(\varTheta\right)$:}~

			            \item \textbf{Técnica de amostragem:}~

			            \item \textbf{Variável:}~

			            \item \textbf{Unidade de medida da variável:}~

			            \item \textbf{Classificação da variável:}~

			            \item \textbf{Escala de medição da variável:}~
		            \end{itemize}

	      \end{enumerate}

	\item Para avaliar uma localidade, o Índice de Desenvolvimento Humano (IDH) foi criado, no final da década de 1990, com o objetivo de oferecer um contraponto a outro índice muito usado até então, o Produto Interno Bruto (PIB), criado no final da década de 1930.Ambos os índices, PIB e IDH, foram reconhecidos internacionalmente e seus criadores receberam o Prêmio Nobel de Economia, naquelas ocasiões. O cálculo do IDH está à cargo do Programa das Nações Unidas para o Desenvolvimento (PNUD) e abrange três dimensões, educação, saúde ou longevidade e renda. A imagem \ref{fig:imagem-da-tabela-1}, a seguir, apresenta alguns dados sobre a Região Metropolitana de Curitiba. Analise essa imagem e escreva algumas observações coerentes com os dados apresentados.
	
	      \begin{figure}[h]
		      \centering
		      \includegraphics[width=\linewidth]{"fig/imagem da tabela 1"}
		      \caption[Imagem da tabela 1]{Imagem da tabela 1 retirada da lista de exercícios 1}
		      \label{fig:imagem-da-tabela-1}
	      \end{figure}

	      \textbf{Resolução:}

	\item Defina a variável "emoção" de modo que ela tenha:

	      \begin{enumerate}[label=\emph{\alph*})]

		      \item escala nominal:

		      \item escala de proporcionalidade (ou escala de razão):

	      \end{enumerate}

	\item Defina a variável "corrente elétrica" de modo que ela tenha:

	      \begin{enumerate}[label=\emph{\alph*})]

		      \item escala nominal:

		      \item escala ordinal:

		      \item escala de proporcionalidade (ou escala de razão)

	      \end{enumerate}

	\item Visite o sítio eletrônico do Programa das Nações Unidas para o Desenvolvimento (PNUD), em <https://www.undp.org>, e o do PNUD
	      Brasil, em <http://www.br.undp.org>. No do PNUD Brasil, explore os links "IDH", "Atlas do Desenvolvimento Humano" e o "Atlas do Desenvolvimento Humano no Brasil". Visite e
	      explore, ainda, o sítio eletrônico do Instituto Paranaense de Desenvolvimento
	      Econômico e Social (IPARDES), em <http://www.ipardes.pr.gov.br>, e o do Instituto
	      Brasileiro de Geografia e Estatística (IBGE), em <https://www.ibge.gov.br/>. No do
	      IBGE, explore o recurso eletrônico "IBGE Países", em <https://paises.ibge.gov.br/\#/>,
	      e o "IBGE Educa", em <https://educa.ibge.gov.br/>. Acesso em 16/03/2020.

	\item Identifique o tipo de amostragem usado em cada caso: se aleatória simples,
	      estratificada, sistemática ou de conveniência:

	      \begin{enumerate}[label=\emph{\alph*})]

		      \item Uma revista baseou suas conclusões sobre um escândalo político, nas respostas
		            daqueles leitores que acessaram a página eletrônica da revista e emitiram suas
		            opiniões a respeito de tal escândalo.

		            \textbf{Resolução:}~

		      \item Uma professora escreveu o nome de cada um de seus 40 alunos, separadamente,
		            em um pedaço de papel; depois, ela misturou esses 40 papéis e extraiu 10 nomes.

		            \textbf{Resolução:}~

		      \item Um operário retira da esteira de produção uma peça a cada meia hora, para inspeção
		            e controle da qualidade.

		            \textbf{Resolução:}~

		      \item Um programa sobre planejamento familiar pesquisou 800 homens e 800 mulheres
		            sobre seus pontos de vista a respeito do uso de pílulas anticoncepcionais e de
		            preservativos masculinos.

		            \textbf{Resolução:}~

		      \item Para o teste de uma vacina, um pesquisador utilizou-se de 250 voluntários.

		            \textbf{Resolução:}~

	      \end{enumerate}

	\item Um reflorestamento, no formato de um retângulo, tem 950 árvores de eucalipto, sendo
	      50 árvores plantadas em linha, no sentido do comprimento do terreno, e 19 árvores
	      plantadas em linha, no sentido da largura do terreno. Todas elas foram plantadas ao mesmo
	      tempo e com o mesmo espaçamento. No período de seu crescimento, todas elas
	      receberam luz, água, adubo, dentre outros, de forma semelhante. Deseja-se selecionar
	      uma amostra de 25 dessas árvores para fazer estudos inferenciais.

	      \begin{enumerate}[label=\emph{\alph*})]

		      \item Como você selecionará essa amostra usando amostragem aleatória simples?

		      \item Como você selecionará essa amostra usando amostragem sistemática?

		      \item Qual amostragem é mais indicada, de modo a respeitar os princípios da aleatoriedade,
		            da representatividade e da economicidade? Justifique sua resposta.

	      \end{enumerate}

	\item Numa escola, há 1125 estudantes, sendo que 642 declaram-se do sexo masculino, 457
	      declaram-se do sexo feminino e 26 declaram-se de outros sexos. Um levantamento estatístico
	      será realizado e necessita de uma amostra probabilística de 120 estudantes, sendo suposto a
	      variável sexo influenciar nos resultados.

	      \begin{enumerate}[label=\emph{\alph*})]

		      \item Qual a técnica de amostragem indicada?


		      \item Calcule o número de estudantes do sexo masculino, o número de estudantes do sexo
		            feminino e o número de estudantes de outros sexos que a amostra deve ter. Apresente o
		            resultado numa tabela, com a coluna da população e a coluna da amostra.

		      \item Como você selecionará esse número de estudantes de cada sexo, para compor a
		            amostra, de modo a respeitar os princípios da aleatoriedade, representatividade e
		            economicidade?

	      \end{enumerate}

	\item Uma instituição de ensino superior possui os seguintes cursos de graduação:
	      Engenharia Mecânica, Engenharia Eletrônica, Engenharia Civil, Arquitetura e Urbanismo,
	      Medicina, Odontologia, Tecnologia em Processos Ambientais, Tecnologia em Radiologia,
	      Licenciatura em Letras, Licenciatura em Física, Licenciatura em Matemática, Licenciatura
	      em História, Bacharelado em Estatística, Bacharelado em Química, Bacharelado em
	      Design, Bacharelado em Administração. Cada curso, alocado num prédio próprio, está
	      situado em locais diferentes de um grande campus universitário. Deseja-se fazer um
	      levantamento estatístico sobre o conhecimento dos estudantes dessa instituição, a respeito
	      da reciclagem do lixo doméstico. Presume-se que os estudantes dos diferentes cursos
	      tenham conhecimento similar sobre o assunto, com exceção dos estudantes de Tecnologia
	      em Processos Ambientais. Como você selecionará uma amostra probabilística dos
	      estudantes dessa instituição, de modo a respeitar os princípios da aleatoriedade, da
	      representatividade e da economicidade?


	\item Um pesquisador enviou, por e-mail, um questionário a um grupo de 5.000 egressos de
	      uma instituição de ensino superior e recebeu o questionário preenchido de 750 deles. Dos
	      750 respondentes, 63\% afirmaram que não conseguiram emprego na respectiva área de
	      formação, no primeiro ano após a formatura. Esse pesquisador divulgou que “63\% dos
	      egressos daquela instituição de ensino levam mais de um ano após sua formatura, para
	      conseguir um emprego em sua área de formação”. Essa divulgação está coerente com a
	      pesquisa realizada? Justifique.


	\item Um levantamento de dados realizado numa universidade brasileira pretende obter e
	      analisar as alturas dos estudantes que respondem ao seguinte item: “Registre sua altura,
	      em polegadas”.

	      \begin{enumerate}[label=\emph{\alph*})]

		      \item Identifique dois problemas nesse levantamento.

		      \item Como você agiria para levantar as alturas dos estudantes, de modo a coletar dados
		            fidedignos?

	      \end{enumerate}

	\item Nos itens abaixo, justifique se a técnica de amostragem usada é tendenciosa ou não:

	      \begin{enumerate}[label=\emph{\alph*})]

		      \item Numa pesquisa sobre o salário médio dos funcionários de empresas de construção
		            civil, em certa empresa sorteada, a pesquisadora deparou-se com uma população
		            heterogênea em relação à variável de estudo: os oito diretores da empresa recebiam
		            salários entre R\$ 9.500,00 a R\$ 11.200,00; os 45 gerentes recebiam de R\$ 4.500,00
		            a R\$ 5.300,00; os 320 técnicos, de R\$ 2.400,00 a 3.850,00; e os 200 atendentes, de
		            R\$ 950,00 a R\$ 1.560,00. Como, nessa empresa, a amostra devia ser de 80
		            funcionários, a pesquisadora sorteou-os, aleatoriamente, pelo processo de
		            cumbuca, ou seja, fez uma amostragem aleatória simples, e calculou o salário médio
		            desses 80 funcionários.

		      \item Para investigar a proporção de operários de uma fábrica favoráveis à mudança do
		            início das atividades, das 7h para às 7h30min, foi decidido entrevistar os 30 primeiros
		            operários que chegassem à fábrica, na quarta-feira.

		      \item Para verificar o efeito do brinde nas vendas de sabão em pó de determinada marca,
		            foi constituída uma amostra com oito supermercados da zona sul e oito
		            supermercados da zona norte de uma metrópole. Nos oito supermercados da zona
		            sul, o produto foi vendido com brinde, enquanto nos outros oito foi vendido sem
		            brinde. No fim de um mês, comparou-se as vendas da zona sul e da zona norte.8

		      \item Para determinar o número médio de pessoas que vivem numa casa, foi planejado
		            visitar 1.000 casas. Como o entrevistador não achou ninguém em 133 das 1.000
		            casas que devia visitar, ele visitou as casas vizinhas a essas 133, completando,
		            assim, a amostra de 1.000 casas.

		      \item Para determinar o número médio de pessoas que vivem numa casa, foi planejado
		            visitar 1.000 casas. Como o entrevistador não achou ninguém em 133 das 1.000
		            casas que devia visitar, ele visitou as casas vizinhas a essas 133, completando,
		            assim, a amostra de 1.000 casas.

	      \end{enumerate}

	\item Os 687 alunos do Ensino Médio, do período noturno, de uma escola estão distribuídos
	      em três séries: 1\textsuperscript{a}, 2\textsuperscript{a} e 3\textsuperscript{a}. Pretende-se fazer um estudo para conhecer os hábitos de fumar
	      desses alunos. Decidiu-se então fazer uma entrevista com 150 alunos. Para constituir uma
	      amostra foram feitas duas propostas. A primeira sugeria escrever os nomes de todos os
	      alunos em cédulas, misturar bem e tirar 150. A segunda sugeria pedir aos professores que
	      indicassem os nomes dos alunos que eles achavam que deveriam fazer parte da amostra.
	      Discuta essas duas propostas e, se for o caso, faça uma terceira.

	\item Assinale V se a sentença for verdadeira e F se for falsa. Justifique as falsas:

	      \begin{enumerate}[label=\emph{\alph*})]


		      \item (~~) Uma das finalidades principais da Estatística Aplicada é conhecer certas
		            características de uma população ou universo, com base nos dados de uma amostra
		            representativa desse universo.


		      \item (~~) Se os objetivos de uma pesquisa não forem elaborados de forma bastante clara, há
		            um grande risco de ela ser inviabilizada a qualquer momento, pois todas as etapas
		            da pesquisa tomam esses objetivos como base.

		      \item (~~) Na elaboração de um questionário para entrevistas é preferível fazer questões abertas,
		            ou seja, aquelas que podem ser respondidas livremente pelo entrevistado, não sendo,
		            portanto, oferecidas alternativas por parte do entrevistador; pois esse tipo de questão é
		            mais fácil de ser elaborada, bem como, mais fácil de serem apuradas as respostas.

		      \item (~~) Pesquisa-piloto é uma pesquisa prévia que deve ser feita numa pequena amostra da
		            população em estudo, antes da coleta geral dos dados, a fim de detectar possíveis
		            falhas nos instrumentos de coleta de dados (formulários, questionários, aparelhos de
		            medição, por exemplo) ou no próprio processo dessa coleta.

		      \item (~~) Os gráficos e tabelas são ferramentas estatísticas usadas para apresentar de forma
		            resumida, objetiva e clara os dados coletados e apurados.

		      \item (~~) Um gráfico bonito e bem apresentado esteticamente é um gráfico confiável.

		      \item (~~) Não se pode dar garantia de que a estimativa tenha valor igual ou mesmo um valor
		            próximo do parâmetro que se pretende estimar. No entanto, se a amostra for
		            suficientemente grande e obtida por meio de técnica de amostragem adequada, a
		            estimativa será próxima do parâmetro, na maioria das vezes. Ainda, amostras
		            sucessivas e criteriosas da mesma população tendem a fornecer estimativas
		            similares entre si e com valores em torno do parâmetro.

		      \item (~~) Num estudo estatístico, tendência, viés ou vício é uma diferença sistemática entre a
		            estatística e o parâmetro que se quer estimar.

		      \item (~~) Do ponto de vista científico, é preferível o estudo cuidadoso de uma amostra do que
		            o estudo rápido de toda a população que se deseja investigar.

		      \item (~~) Em Estatística, a preocupação central é com o estudo de fenômenos coletivamente
		            típicos, ou seja, aqueles que apresentam regularidade na massa de observações, não
		            necessariamente nos casos isolados. Dessa forma, estudos de caso obtidos de
		            amostras muito pequenas não servem de base para se estimar o comportamento da
		            população de onde esses casos provieram, pois a probabilidade de obter uma
		            estimativa que se desvia muito do parâmetro aumenta quando a amostra é pequena.

		      \item (~~) Do ponto de vista científico, geralmente é mais relevante o tamanho absoluto da amostra
		            do que a porcentagem que ela representa da população, pois não é o tamanho, apenas,
		            que determina se a amostra é representativa da população de onde ela proveio.

		      \item (~~) Na maioria das pesquisas estatísticas, a pessoa a ser entrevistada geralmente não
		            sabe o uso que se fará das informações que ela fornecerá. Essa prática está
		            eticamente correta, pois, afinal, as conclusões proporcionadas pela pesquisa
		            importam, sobretudo, para aqueles que pagaram para obter as informações.

		      \item (~~) Nas pesquisas com seres humanos, como os testes de medicamentos, é permitido
		            que o pesquisador investigue suas hipóteses científicas em grupos de pessoas que
		            não saibam que estão participando de uma pesquisa. Isso porque a investigação
		            somente a partir de voluntários não garantiria ao pesquisador conclusões confiáveis
		            para fazer inferências, isto é, estender as conclusões retiradas com base na amostra
		            para toda a população de onde essa amostra proveio.

	      \end{enumerate}

\end{enumerate}







% PARTE DA PREPARAÇÃO DA PESQUISA
% ----------------------------------------------------------
%\part{Preparação da pesquisa}
%\input{cap02}
%

% PARTE DOS REFERENCIAIS TEÓRICOS
% ----------------------------------------------------------
%\part{Referenciais teóricos}
%\input{cap03}

% PARTE DOS RESULTADOS
% ----------------------------------------------------------
%\part{Resultados}
%\input{cap05}

% Finaliza a parte no bookmark do PDF
% para que se inicie o bookmark na raiz
% e adiciona espaço de parte no Sumário
% ----------------------------------------------------------
%\phantompart

% ---
% Conclusão (outro exemplo de capítulo sem numeração e presente no sumário)
% ---
%\chapter*[Conclusão]{Conclusão}
%\addcontentsline{toc}{chapter}{Conclusão}
% ---
%\input{cap06}

% ELEMENTOS PÓS-TEXTUAIS
% ----------------------------------------------------------
\postextual

% Ajuste vertical do titulo de referencias no sumário
% ----------------------------------------------------------
\addtocontents{toc}{\vspace{-24pt}}

% Referências bibliográficas
% ----------------------------------------------------------
%\bibliography{referencias}

\printbibliography[heading=bay]
% ----------------------------------------------------------

% Ajuste vertical dos titulos dos capitulos postextuais
% ----------------------------------------------------------
\addtocontents{toc}{\vspace{-12pt}}

% Glossário
% ----------------------------------------------------------
% Consulte o manual da classe abntex2 para orientações sobre o glossário.
%
%\glossary

% Apêndices
% ----------------------------------------------------------
\ifthenelse{\equal{\terApendice}{Sim}}
{\begin{apendicesenv}

        % Numeração arábica para os apêndices
        % --------------------------------------------------
        \renewcommand{\thechapter}{\arabic{chapter}}
        % Imprime uma página indicando o início dos apêndices
        % \partapendices

        % Existem várias formas de se colocar anexos.
        % O exemplo abaixo coloca 2 apêndices denominados de 
        % DESENVOLVIMENTO DETALHADO DA PINTURA e 
        % ESCOLHA DO MATERIAL DE IMPRESSÃO:
        % ---
        % --- insere um capítulo que é tratado como um apêndice
        %\chapter{DESENVOLVIMENTO DETALHADO DA PINTURA}
        % 
        %\lipsum[29] % gera um parágrafo
        %
        % --- insere um capítulo que é tratado como um apêndice
        %\chapter{ESCOLHA DO MATERIAL DE IMPRESSÃO}
        % 
        %\lipsum[30] % gera um parágrafo

        % --- Insere o texto do arquivo ap01.tex
        % 
        % --- O conteúdo do arquivo pode ser vários anexos ou um único apêndices.
        %     A vantagem de se utilizar este procedimento é de suprimi-lo
        %     das compilações enquanto se processa o resto do documento.

           % --- insere um capítulo que é tratado como um apêndice
   \label{ap:ap01}
   \poschap{ESCOLHA DO MATERIAL - colocado no apendice}
    
   \lipsum[30] % gera um parágrafo
   \section*{Testes se\c{C}\~aO}

    \lipsum[22] % gera um parágrafo
	
	\poschap{ESCOLHA DO MATERIAL DE IMPRESSÃO- colocado no apendice}
    \lipsum[32] % gera um parágrafo


\end{apendicesenv}
}{}


% Anexos
% ----------------------------------------------------------
\ifthenelse{\equal{\terAnexo}{Sim}}{
\begin{anexosenv}

        % Numeração arábica para os apêndices
        % --------------------------------------------------
        \renewcommand{\thechapter}{\arabic{chapter}}
        % --- Imprime uma página indicando o início dos anexos
        % \partanexos

        % Existem várias formas de se colocar anexos.
        % O exemplo abaixo coloca 2 anexos denominados de 
        % TABELA DE VALORES e GRÁFICOS DE BALANCEMANTO:
        % ---
        % --- insere um capítulo que é tratado como um anexo
        %\chapter{TABELAS DE VALORES}
        % 
        %\lipsum[31] % gera um parágrafo
        %
        % --- insere um capítulo que é tratado como um anexo
        %\chapter{GRÁFICOS DE BALANCEAMENTO}
        % 
        %\lipsum[32] % gera um parágrafo

        % --- Insere o texto do arquivo ax01.tex
        % 
        % --- O conteúdo do arquivo pode ser vários anexos ou um único anexo.
        %     A vantagem de se utilizar este procedimento é de suprimi-lo
        %     das compilações enquanto se processa o resto do documento.

            % --- insere um capítulo que é tratado como um apêndice
   \poschap{anexando ESCOLHA DO MATERIAL}
    
   \lipsum[30] % gera um parágrafo
    \section*{anexando testes secao}
    

\poschap{anexando ESCOLHA DO MATERIAL DE IMPRESSÃO}
    
   \lipsum[32] % gera um parágrafo
 
\end{anexosenv}
}{}

% INDICE REMISSIVO
%---------------------------------------------------------------------
\ifthenelse{\equal{\terIndiceR}{Sim}}{
\phantompart
\printindex
}{}

\end{document}
