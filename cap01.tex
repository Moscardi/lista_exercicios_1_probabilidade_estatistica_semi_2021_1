\chapter[INTRODUÇÃO]{INTRODUÇÃO}
%---------------------------------------------------------------------------------------


\cite{ISO5122:1979}

\textcite{ISO5122:1979}

\begin{itemize}
 \item para inserir uma sigla: 
    \verb|\criarsigla|\{ABNT\}{Associa\c{c}\~ao Brasileira de Normas T\'ecnicas}:

    \criarsigla{ABNT}{Associa\c{c}\~ao Brasileira de Normas T\'ecnicas}

\item para inserir s\'imbolo: 
    \verb|\criarsimbolo{$ \Gamma $}{Letra grega Gama}|
  
    $ \Gamma $ \criarsimbolo{$ \Gamma $}{Letra grega Gama}

\end{itemize}

% Usado para testar o formato uppercase dos t\'itulos 
% em maiusculas nas respectivas listas
%---------------------------------------------------------------------------------------
\begin{table}[!ht]
 \centering
 \par\caption{TesTANDO TABELAS}

\begin{tabular}{c|c|c}
 teste1&teste1&teste1\\\hline\hline
  1&2&3\\\hline
 \end{tabular}
 \label{tab:tab01}
\end{table}



\tabela{tabela teste} %1 Título da tabela
{
\begin{tabular}{c|c|c|c|c|c}
 teste1&teste2&teste3&teste1&teste2&teste3\\\hline\hline
  1&2&3&4&5&6\\\hline
 \end{tabular}
} %2 Tabela
{\textcite{ISO5122:1979}}%3 Fonte da tabela
{teste1} %4 Label da tabela tab:teste1
{ Nota de teste } %5 Nota da tabela
{testando as figuras e tabelas, fda} %6 Legenda da tabela


\figura
{TESTE DE FIGURAS 2} %1 Legenda
{.55} %2  % da largura da área de texto
{fig/figure} %3 localização da figura
{\textcite[1]{abntex2modelo}} %4 fonte da figura
{teste} %5 etiqueta
{\url{https://goo.gl/EKFRak} TESTE DE FIGURAS 2 TESTE DE FIGURAS 2 TESTE DE FIGURAS 2 TESTE DE FIGURAS 2 TESTE DE FIGURAS 2 TESTE DE FIGURAS 2 TESTE DE FIGURAS 2 TESTE DE FIGURAS 2 TESTE DE FIGURAS 2 TESTE DE FIGURAS 2 TESTE DE FIGURAS 2} %6 Nota da figura
{} %7 Legenda da figura

\figura
{TESTE DE FIGURAS 3} % Legenda
{.65} % % da largura da área de texto
{fig/tipog} % localização da figura
{o Autor (2017)} % fonte da figura
{tipo1} % etiqueta
{}
{}

\figura
{Figura original} % Legenda
{.35} % % da largura da área de texto
{fig/fig} % localização da figura
{\textcite{luminaria01}} % fonte da figura
{tipo2} % etiqueta
{}
{}

\figurac
{Figura aparada} % Legenda
{.35} % % da largura da área de texto
{fig/fig} % localização da figura
{\textcite{luminaria01}} % fonte da figura
{tipo3} % etiqueta =>  fig:tipo3
{notinha}   % Nota
{legendonha} % legenda
{70} % laterais mm
{80} % superior e inferior mm


A \autoref{fig:teste} apresenta o seguinte detalhe que deve ser observado em pacientes com Wordnite

%---------------------------------------------------------------------------------------
\chapter{REVISÃO TEÓRICA}

\textcite[3.1-3.2]{abntex2modelo}

\textcite{luminaria01}

\textcite{luminaria01}


\textcite{luminaria01}



\textcite{luminaria01}

\textcite{ISO5122:1979}

\section{Modelo}

\subsection{Modelo simplificado}
\newpage
\textcite[3.1-3.2]{abntex2modelo}

\subsubsection{Modelo simplificadíssimo}
